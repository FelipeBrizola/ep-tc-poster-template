\documentclass[a0,portrait]{lab-poster}

\usepackage[brazil]{babel}    % Configuração de Linguagem (comentar para inglês)
% \renewcommand{\tablename}{Tabela}
% \renewcommand{\figurename}{Figura}

% \newcommand\itemadjust{\itemsep.5em \parskip0pt \parsep0pt}

\title{Seu Título Aqui}
\author{Nome do Aluno 1, Nome do Aluno2}
\major{Ciência da Computação}
\advisor{Nome do Orientador}

\themecolor{NavyBlue}
\unilogo{fig/pucrs-logo.pdf}
\lablogo{fig/facin-logo.pdf}

\begin{document}
\maketitle

%---------------------------------------------------------------

\begin{multicols}{2} 
%---------------------------------------------------------------
%	MOTIVAÇÃO
%---------------------------------------------------------------
% \color{NavyBlue}
\section*{Seção 1}
% \color{Black}
\Large
\justifying

\begin{itemize}
	\item Ponto 1;
	\item Ponto 2;
	\item 
\end{itemize}

%---------------------------------------------------------------
%   AGENTES
%---------------------------------------------------------------
\section*{\huge Seção 2}

\begin{itemize}
	\item Idéia 1; e
	\item Idéia 2.
\end{itemize}

%---------------------------------------------------------------
%	AgentSpeak(L)
%---------------------------------------------------------------
\section*{\huge Seção 3}

\begin{itemize}
	\item É uma linguagem abstrata de programação orientada a agentes baseada na arquitetura BDI;
	\item Os programas desenvolvidos utilizando a linguagem, como exemplificado no Código~\ref{alg:exemplo-hello-world}, são especificados por um conjunto de crenças, planos, eventos ativadores e ações que o agente executa no ambiente;
	%\item Rao \cite{article:Rao:1996} introduz as noções básicas para especificação desses conjuntos a partir das definições apresentadas na Tabela~\ref{tab:sintaxe-agentsepak}.
	\item As definições da Tabela~\ref{tab:sintaxe-agentsepak} são utilizadas para especificar estes conjuntos.
\end{itemize}

\vspace{13mm}


\begin{itemize}
	\item Mais idéias...:
	\begin{enumerate}
		% [leftmargin=2em]\itemadjust
		\item Idéia 1;
		\item Idéia 2; 
		\item Idéia 3; e
		\item Idéia 4.
	\end{enumerate}	
	\item A Figura~\ref{fig:minhafigura} ilustra isto.
\end{itemize}
\vspace{13mm}

\begin{center}
	%\includegraphics[width=0.99\linewidth]{fig/minha-figura.pdf}
	\Huge Figura Aqui (include comentado)
	\captionof{figure}{Exemplo de Figura.}
	\label{fig:minhafigura}
\end{center}	


\section*{Seção 3}

\begin{itemize}
	\item Mais idéias...:
	\begin{enumerate}
		% [leftmargin=2em]\itemadjust
		\item Idéia 1;
		\item Idéia 2; 
		\item Idéia 3; e
		\item Idéia 4.
	\end{enumerate}	
	\item A Figura~\ref{fig:minhafigura} ilustra isto.
\end{itemize}


\vspace{13mm}
\noindent\begin{minipage}{.235\textwidth}
	\begin{minipage}{\textwidth}
		\lstset{style=codeStyle}
		\begin{lstlisting}[language=Prolog, label={alg:exemplo-hello-world}, caption={Exemplo de programa em AgentSpeak(L).}]
		/* Agent helloWorld */
		/* Initial beliefs and rules */
		
		/* Initial goals */			
		!start.
		/* Plans */
		+!start : true <- aloha; ?continue(true);      !run (agentspeak).
		+!run(A) : true <- mahalo(A).
		\end{lstlisting}
	\end{minipage}\hfill
	\vspace{7mm}
	
	\begin{minipage}{\textwidth}
		\lstset{style=codeStyle}
		\begin{lstlisting}[language=Prolog, label={alg:exemplo-projeto-hello-world}, caption={Exemplo de projeto do AgentSpeak(Py).}]
		/* Project Name */
		helloWorld:
		// List of agents
		agents = [helloWorld]
		environment = HelloWorldEnv
		\end{lstlisting}
	\end{minipage}\hfill
\end{minipage}\hfill
\begin{minipage}{.235\textwidth}
	\lstset{style=codeStyle}
	\begin{lstlisting}[language=Python, label={alg:exemplo-environment}, caption={Exemplo da descrição do ambiente em Python.}]
	from environment import *
	lt_continue = parse_literal('continue(true)')
	
	class HelloWorldEnv(Environment):
		def __init__(self):
			Environment.__init__(self)
		
		def execute_action(self, agent_name, action):
			self.clear_perceptions()
			getattr(self, action.functor)(list(action.args))
		
		def aloha(self, *args):
			self.add_percept(lt_continue)
			print('Aloha HelloWorldEnv!')
		
		def mahalo(self, *args):
			print('Mahaloing with %s!' % ", ".join(map(str, *args)))
	\end{lstlisting}
\end{minipage}

\section*{\huge Seção 4}
      
Tamanho atual das fontes:
\begin{itemize}
	\item {\tiny \verb|\tiny| Font Size: \showfontsize}
	\item {\scriptsize \verb|\scriptsize| Font Size: \showfontsize}
	\item {\footnotesize \verb|\footnotesize| Font Size: \showfontsize}
	\item {\small \verb|\small| Font Size: \showfontsize}
	\item {\normalsize \verb|\normalsize| Font Size: \showfontsize}
	\item {\large \verb|\large| Font Size: \showfontsize}
	\item {\Large \verb|\Large| Font Size: \showfontsize}
	\item {\LARGE \verb|\LARGE| Font Size: \showfontsize}
	\item {\huge \verb|\huge| Font Size: \showfontsize}
	\item {\Huge \verb|\Huge| Font Size: \showfontsize}
	\item {\veryHuge \verb|\veryHuge| Font Size: \showfontsize}
\end{itemize}


%---------------------------------------------------------------
%	REFERENCES
%---------------------------------------------------------------
% Descomentar abaixo caso queira usar referências bibliográficas no poster
%\vspace{-10mm}
%\large
%\color{NavyBlue}
%\color{Black}
%\raggedright
%\bibliographystyle{plain}
%\bibliography{poster}

\end{multicols}

%----------------------------------------------------------------------------------------
\end{document}